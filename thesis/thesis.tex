% !TeX root = thesis.tex
% !TeX spellcheck = es_ES
% !TeX encoding = UTF-8
% SET PROGRAM LIKE THIS: TeX program = latexmk pdflatex -synctex=1 -pdf -interaction=nonstopmode -file-line-error %.tex
% SET PROGRAM LIKE THIS: TeX TXS-program:pdflatex = pdflatex -synctex=1 -interaction=nonstopmode

\documentclass[12pt, american, spanish]{../packages/masterfotonicaucm}

\usepackage{graphicx}
\graphicspath{{./images}{./tikzfigures}{./}{../images}{../tikzfigures}{../}}

% TO CHANGE TO bibtex, although we recommend keeping biblatex
% \RequirePackage{bibtex}
% bibtex
% \bibliographystyle{elsarticle-num}

% https://www.overleaf.com/learn/latex/Biblatex_citation_styles
% biblatex
\usepackage[
    backend=biber,
    style=numeric,
    sortlocale=en_US,
    natbib=true,
    % sort by order of appeareance
    sorting=none,
    % sort citation keys
    sortcites,
    url=true,
    doi=true,
    eprint=true
]{biblatex}
\addbibresource{../references.bib}
\addbibresource{../mendeley.bib}

\hypersetup{
    linkcolor={red!50!black},
    citecolor={blue!50!black},
    urlcolor={blue!80!black}
    % Do hyperlinks have color or not:
    % colorlinks=false,
    colorlinks=true,
}

\title{%
    Titulo de la tesis
}
\newcommand{\theenglishtitle}{English title of your thesis}

\newcommand{\thetutor}{
   Tu tutor %(50\%)
}

\newcommand{\thetutortwo}{
    Segundo Tutor%(50\%)
}
\author{Nombre LastName LastName2}

\usepackage{lipsum}

\begin{document}
\pagenumbering{gobble}

% %%%%%%Portada%%%%%%%
\begin{titlepage}
    \begin{center}


        \textbf{\Large{}UNIVERSIDAD COMPLUTENSE DE MADRID}
        \vspace{0.5cm}

        \textbf{\Large{}FACULTAD DE CIENCIAS FÍSICAS}
        \vspace{1cm}

        { \textbf{\large{}Máster en Nuevas Tecnologías Electrónicas y Fotónicas}} %
        \vspace{0.8cm}

        %%%%Logo Complutense%%%%%
        % {\includesvg[height=6cm]{ucm-logo.svg}}
        {\includegraphics[height=6cm]{ucm-logo.pdf}}
        \vspace{0.8cm}

        \textbf{\Large{}TRABAJO DE FIN DE MÁSTER} %
        \vspace{2cm}

        \textbf{\Large{}\thetitle}%

        \textsc{ \theenglishtitle}%

        \vspace{20mm}

        {\bfseries \LARGE{}\theauthor}%
        \vspace{10mm}

        \textbf{\textsc{\Large{}\thetutor}}%
        \vspace{5mm}

        \textbf{\textsc{\Large{}\thetutortwo}}%
        \vspace{15mm}

        {\large{}Curso académico 2024}%
        \vspace{5mm}
    \end{center}
\end{titlepage}%
\cleardoublepage%


\pagestyle{plain}

\pagenumbering{roman}

\section*{{\large{}Resumen: }}

Esto es una prueba para probar el formato del Resumen.

\section*{{\large{}Abstract: }}

This is a test to prove the abstract's layout.

\vfill

\section*{{\large{}Palabras clave: }}

Palabra 1, palabra2

\section*{{\large{}Keywords: }}

Palabra 1, palabra2

\cleardoublepage%

\section*{{\large{}Agradecimientos }}

No es necesario pero se puede agradecer.

\cleardoublepage%
\tableofcontents%
\cleardoublepage%


\pagestyle{plain}

\pagenumbering{arabic}

\section{Introduction}

Citamos con este comando~\cite{kour2014fast}.

\lipsum[1-2]

\subsection{Subsection}

\lipsum[3-4]

\section{Conclusions}

Conclusion citando~\cite{hadash2018estimate,kour2014real}

\lipsum[5-6]


\addcontentsline{toc}{section}{\refname}
% bibtex
% \bibliography{references.bib}
% biblatex
\printbibliography[]


\cleardoublepage%

\appendix
\pagenumbering{roman}

\appendixpage%
\addappheadtotoc%

\section{Software usado}


\section{Otro apendice}

\end{document}
